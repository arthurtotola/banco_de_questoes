%Data da última atilaização: 19/10/2016
\chapter{Equações do Primeiro Grau.} \label{cap1}
		\section{Atividades Básicas}				
			\begin{list}{\textbf{Questão \arabic{quest}.}}{\usecounter{quest}}
%define a margem da lista.	
%\setlength{\labelwidth}{-2mm} \setlength{\parsep}{0mm}
%\setlength{\topsep}{0mm} \setlength{\leftmargin}{-2mm}
\renewcommand{\labelenumi}{(\alph{enumi})}
				
				\item Dada a equação $7x-3+x=5-2x$, responda.
					\setlength{\columnsep}{5pt}%define o espaçamento entre as colunas								
					\begin{multicols}{2}					
					\begin{enumerate}
						\item Qual é o primeiro membro?
						\item Qual é o segundo membro?
						\item Quais são os termos do primeiro termo?
						\item Quais são os termos do segundo termo?
					\end{enumerate}
					\end{multicols}
						
				\item Qual é o número que colocado no lugar de $x$, torna verdadeira as sentenças?
					\begin{multicols}{4}
					\begin{enumerate}
						\item $x+9=13$
						\item $x-7=10$
						\item $5x-1=9$
						\item $x-3=8$
					\end{enumerate}
					\end{multicols}
				\item Resolva as equações:
					\begin{multicols}{4}
					\begin{enumerate}
						\item $x+5=8$
						\item $x-4=3$
						\item $x+6=5$
						\item $x-7=7$
						\item $x+9=-1$
						\item $x+28=11$
						\item $x-109=5$
						\item $x-39=-79$
						\item $10=x+8$
						\item $15=x+20$
						\item $4=x-10$
						\item $7=x+8$
						\item $0=x+12$
						\item $-3=x+10$
					\end{enumerate}
					\end{multicols}
				\item Resolva:
					\begin{multicols}{4}
					\begin{enumerate}
						\item $-x=9$
						\item $-x=-2$
						\item $-7x=14$
						\item $-3x=10$
						\item $-5x=-12$
						\item $-4x=8$
						\item $-3x=-9$
						\item $-5x=15$
						\item $-2x=-10$
						\item $15=-3x$
						\item $-40=-5x$
										    
				    \end{enumerate}	
				    \end{multicols}								
				
				
				\item Resolva as seguintes equações:
					\begin{multicols}{4}					
					\begin{enumerate}
						\item $3x=15$
						\item $2x=14$
						\item $4x=-12$
						\item $7x=-21$
						\item $13x=13$
						\item $9x=-9$
						\item $25x=0$
						\item $35x=-105$
						\item $4x=1$
						\item $36x=12$
						\item $21=3x$
						\item $84=6x$
					\end{enumerate}							
					\end{multicols}
				\item Resolva as equações:
					\begin{multicols}{3}
					\begin{enumerate}
						\item $\displaystyle\frac{x}{3}=7$
						\item $\displaystyle\frac{x}{4}=-3$
						\item $\displaystyle\frac{2x}{5}=4$
						\item $\displaystyle\frac{2x}{3}=-10$
						\item $\displaystyle\frac{3x}{4}=30$
						\item $\displaystyle\frac{2x}{5}=-18$
					\end{enumerate}
					\end{multicols}				
							
				\item Resolva as equações a seguir:
					\begin{multicols}{2}
					\begin{enumerate}
						\item $18x - 43 = 65$
						\item $23x - 16 = 14 - 17x$
						\item $10y - 5 (1 + y) = 3 (2y - 2) - 20$
						\item $x(x + 4) + x(x + 2) = 2x^2 + 12$
						\item $\displaystyle\frac{(x - 5)}{10} + \frac{(1 - 2x)}{5} = \frac{(3-x)}{4}$                                 
						\item $4x (x + 6) + x^2 = 5x^2$
					\end{enumerate}				
					\end{multicols}
				\item Verifique se $1$ é a raiz da equação $\displaystyle{4x + \frac{1}{2} = \frac{9}{2}}$
						
%marcador				
       			\item Determine um número real $a$ para que as expressões $\frac{3a + 6}{8}$ e $\frac{2a + 10}{6}$ sejam iguais.
       			\item Resolver as seguintes equações (na incógnita $x$):
					\begin{multicols}{2}					
					\begin{enumerate}
		 			\item $\displaystyle\frac{5}{x} - 2 = \frac{1}{4} (x\neq 0)$
		 			\item $3bx + 6bc = 7bx + 3bc$
		 			\end{enumerate} 
		 			\end{multicols}
		 		\item Determine o valor de $x$ na equação a seguir aplicando as técnicas resolutivas.
		 			\begin{multicols}{2}
		 			\begin{enumerate}
		 				\item $3 - 2(x + 3) = x - 18$
		 				\item $50 + (3x - 4) = 2  (3x - 4) + 26$
		 			\end{enumerate}
		 			\end{multicols}
		 		\item Qual é a raiz da equação $7x - 2 = -4x + 5$?
		 		\item Resolva as Equações em $\mathbb{R}$.
		 			\begin{multicols}{2}
		 			\begin{enumerate}
		 				 \item $2x + 6 = x + 18$
		 				 \item $5x - 3 = 2x + 9$
		 				 \item $3(2x - 3) + 2(x + 1) = 3x + 18$
		 				 \item $2x + 3(x - 5) = 4x + 9$
		 				 \item $2(x + 1) - 3(2x - 5) = 6x - 3$
		 				 \item $3x - 5 = x - 2$
		 				 \item $3x - 5 = 13$
		 				 \item $3x + 5 = 2$
		 				 \item $x - (2x - 1) = 23$
		 				 \item $2x - (x - 1) = 5 - (x - 3)$   
		 			\end{enumerate}
		 			\end{multicols}
		 		\item O valor numérico da expressão $2x^2 + 8$, para $x$ igual a $-3$ é:		
		 			\setlength{\columnseprule}{0pt}
					\begin{multicols}{4}					
					\begin{enumerate}
						\item 17
						\item 18
						\item 26   
						\item 34
					\end{enumerate}
					\end{multicols}
				\item Resolva as equações.
					\setlength{\columnseprule}{0pt}
					\setlength{\columnsep}{20pt}					
					\begin{multicols}{4}
					\begin{enumerate}
						\item $2x - 3 = 15                                                   $
						\item $4y = 30 - 18                                                 $
						\item $5z - 6 = z + 14                                              $
						\item $m + 4 = 20 $
					\end{enumerate}
					\end{multicols}
				\item Resolva– as equações
					\begin{multicols}{2}
					\begin{enumerate}
						\item $4x-1=3(x-1)$
						\item $3(x-2)=2x-4$
						\item $2(x-1)=3x-4$
						\item– $3(x-1)-7=15$
						\item $7(x-4)=2x-3$
						\item $3(x-2)=4(3-x)$
						\item $3(3x-1)=2(3x+2)$
						\item $7(x-2)=5(x+3)$
						\item $3(2x-1)=-2(x+3)$
						\item $5x-3(x+2)=15$
						\item $2x+3x+9=8(6-x)$
						\item $4(x-10)-2(x-5)=0$
						\item $3(2x+3)-4(x-1)=3$
						\item $7(x-1)-2(x-5)=x-5$
						\item $2(3-x)=3(x-4)+15$
						\item $3(5-x)-3(1-2x)=42$
						\item $(4x+6)-2x=(x-6)+10+14$
						\item $(x-3)-(x+2)+2(x-1)-5=0$
						\item $3x-2(4x-3)=2-3(x-1)$
						\item $3(x-1)-(x-3)+5(x-2)=18$
						\item $5(x-3)-4(x+2)=2+3(1-2x)$
					\end{enumerate}
					\end{multicols}
				\item Resolva as seguintes equações:
					\begin{multicols}{3}
					\begin{enumerate}
						\item $\displaystyle\frac{x}{4}-\frac{x}{6}=3$
						\item $\displaystyle\frac{3x}{4}-\frac{x}{3}=5$
						\item $\displaystyle\frac{x}{5}-1=9$
						\item $\displaystyle\frac{x}{3}-5=0$
						\item $\displaystyle\frac{x}{2}+\frac{3x}{5}=6$
						\item $\displaystyle\frac{x}{5}+\frac{x}{2}=\frac{7}{10}$
						\item $5x-10=\displaystyle\frac{x+1}{2}$
						\item $\displaystyle\frac{8x-1}{x}-2x=3$
						\item $\displaystyle\frac{2x-7}{5}=\frac{x+2}{3}$
						\item $\displaystyle\frac{5x}{2}=2x+\frac{x-2}{3}$
						\item $\displaystyle\frac{x-3}{4}-\frac{2x-1}{5}=5$
						\item $\displaystyle\frac{x-1}{2}+\frac{x-3}{3}=6$
						\item $\displaystyle\frac{5x-7}{2}=\frac{1}{2}+x$
						\item $\displaystyle\frac{2x-1}{3}=x-\frac{x-1}{5}$
						\item $\displaystyle\frac{x}{4}+\frac{3x-2}{2}=\frac{x-3}{2}$
						\item $\displaystyle\frac{2(x-1)}{3}=\frac{3x+6}{5}$
						\item $\displaystyle\frac{3(x-5)}{6}+\frac{2x}{4}=7$
						\item $\displaystyle\frac{x}{5}-2=\frac{5(x-3)}{4}$
					\end{enumerate}
					\end{multicols}					
			\end{list}	
		\section{Problemas Envolvendo Equações}
			\begin{list}{\textbf{Questão \arabic{quest}.}}{\usecounter{quest}}
%define a margem da lista.	
%\setlength{\labelwidth}{-2mm} \setlength{\parsep}{0mm}
%\setlength{\topsep}{0mm} \setlength{\leftmargin}{-2mm}
\renewcommand{\labelenumi}{(\alph{enumi})}

				\item Resolva cada situação problema a seguir, utilizando equações do primeiro grau com uma variável:
					\begin{enumerate}
						\item O triplo de um número somado a quatro é igual a vinte e cinco. Qual é este número?
						\item O quíntuplo do número de meninas da 7$^{a}$A menos cinco é igual a 25. Quantas são as meninas da 7$^{a}$ A

						\item A diferença entre o triplo de um número e 90 é igual a esse número somado com 48. Que número é esse?

						\item Um número menos 12 é igual a 3/4 do mesmo número. Qual é esse número?

						\item O triplo de um número menos 40 é igual a sua metade mais 20. Que número é esse?

						\item A metade de um número mais 10 e mais a sua terça parte é igual ao próprio número. Que número é esse?

						\item Sabe-se que 3/5 da idade de Jurandir menos 15 é igual a 9. Qual é a idade de Jurandir?

						\item Um número é o triplo do outro. Somando os dois, obtemos 84. Quais são esses números?

						\item A idade de um pai é o triplo da idade de seu filho. Calcule essas idades, sabendo que juntos eles possuem 72 anos.

						\item Somando 5 anos ao dobro da idade de Sônia, obtemos 35 anos. Qual é a idade da Sônia?						

						\item Um número tem 4 unidades a mais que o outro. A soma deles é 150. Quais são estes números?						

						\item Tenho 9 anos a mais que meu irmão e juntos temos 79 anos. Quantos anos eu tenho?

						\item As idades de 3 irmão somam 99 anos. Sabendo-se que o mais jovem tem 1/3 da idade do mais velho e o 2$^{o}$ irmão tem a metade da idade do mais velho, determine a idade do mais velho.

						\item Numa escola 1/3 dos alunos são meninos e 120 são meninas. Quantos alunos há nesta escola?

						\item Numa caixa há bolas brancas e bolas pretas num total de 360. Se o número de bolas brancas é o quádruplo do número de bolas pretas, qual é o número de bolas brancas?

						\item Diminuindo-se 6 anos da idade de minha filha obtém-se os 3/5 de sua idade. Qual é a idade de minha filha?

						\item A soma de três números é 150. O segundo é o triplo do primeiro e o terceiro tem 10 unidades a mais do que o segundo. Quais são estes números?

						\item A quantidade de figurinhas que os irmão Pedro, João e Marcos possuem, somam 142. João tem o quádruplo de figurinhas de Pedro e Marcos o triplo das de João e mais seis figurinhas. Quantas figurinhas tem cada um?

						\item A soma das idades de três irmãos é 28 anos. Quando o segundo nasceu, o primeiro tinha três anos, e quando o terceiro nasceu, o segundo tinha 2 anos. Qual é a idade atual de cada um?

						\item Num campeonato de futebol, as três primeiras equipes classificadas A, B e C,  marcaram 115 gols. A equipe A marcou 12 gols a mais que a equipe C e oito gols a mais que a equipe B. Quantos gols marcou cada equipe?

						\item Jair e Edson têm juntos 35 mil reais. Jair tem a mais que Edson 6 mil reais. Quanto tem cada um?

						\item Ary e Rui têm juntos 840 reais. A quantia de Ary é igual a 3/4 da quantia de Rui. Quantos reais Rui tem?

						\item Numa partida de basquetebol, as duas equipes fizeram um total de 145 pontos. A equipe A fez o dobro de pontos menos cinco, que a equipe B. Quantos pontos marcou cada equipe? 
					\end{enumerate}
				\item O dobro de um número, aumentado de 15, é igual a 49. Qual é esse número?
				\item A soma de um número com o seu triplo é igual a 48. Qual é esse número?
				\item A idade de um pai é igual ao triplo da idade de seu filho. Calcule essas idades, sabendo que juntos têm 60 anos?
				\item Somando 5 anos ao dobro da idade de Sônia, obtemos 35 anos. Qual é a idade de Sônia?
				\item O dobro de um número, diminuído de 4, é igual a esse número aumentado de 1. Qual é esse número?
				\item O triplo de um número, mais dois, é igual ao próprio número menos quatro. Qual é esse número?
				\item O quádruplo de um número, diminuído de 10, é igual ao dobro desse número, aumentado de 2. Qual é esse número?
				\item O triplo de um número, menos 25, é igual ao próprio número, mais 55. Qual é esse número?				
				\item Um número somado com sua quarta parte é igual a 80. Qual é esse número?
				\item Um número mais a sua metade é igual a 15. Qual é esse número?
				\item A diferença entre um número e sua quinta parte é igual a 32. Qual é esse número?
				\item O triplo de um número é igual a sua metade mais 10. Qual é esse número?
				\item O dobro de um número, menos 10, é igual à sua metade, mais 50. Qual é esse número?
				\item A diferença entre o triplo de um número e a metade desse número é 35. Qual é esse número?
				\item Subtraindo 5 da terça parte de um número, obtém-se o resultado 15. Qual é esse número?
				\item A metade dos objetos de uma caixa mais a terça parte desses objetos é igual a 25. Quantos objetos há na caixa?
				\item Em uma fábrica, um terço dos empregados são estrangeiros e 72 empregados são brasileiros. Quantos são os empregados da fábrica?
				\item Flávia e Sílvia têm juntas 21 anos. A idade de Sílvia é três quartos da idade de Flávia. Qual a idade de cada uma?
				\item A soma das idades de Carlos e Mário é 40 anos. A idade de Carlos é três quintos da idade de Mário. Qual a idade de Mário?
				\item A diferença entre um número e os seus dois quintos é igual a trinta e seis. Qual é esse número?
				\item A diferença entre os dois terços de um número e sua metade é igual a seis. Qual é esse número?
				\item Os três quintos de um número aumentados de doze são iguais aos cinco sétimos desse número. Qual é esse número?
				\item Dois quintos do meu salário são reservados para o aluguel e a metade é gasta com alimentação, restando ainda R\$ 45,00 para gastos diversos. Qual é o meu salário?
				\item Lúcio comprou uma camisa que foi paga em 3 prestações. Na 1$^{\underline{a}}$ prestação, ele pagou a metade do valor da camisa, na 2$^{\underline{a}}$ prestação, a terça parte e na última, R\$ 2,00. Quanto ele pagou pela camisa?
				\item Achar um número, sabendo-se que a soma de seus quocientes por 2, por 3 e por 5 é 124.
				\item Um número tem 6 unidades a mais que outro. A soma deles é 76. Quais são esses números?
				\item Um número tem 4 unidades a mais que o outro. A soma deles é 150. Quais são esses números?
				\item Fábia tem cinco anos a mais que Marcela. A soma da idade de ambas é igual a 39 anos. Qual é a idade de cada uma?
				\item Marcos e Plínio tem juntos R\$ 350,00. Marcos tem a mais que Plínio R\$ 60,00. Quanto tem cada um?
				\item Tenho nove anos a mais que meu irmão, e juntos temos 79 anos. Quantos anos eu tenho?
				\item O perímetro de um retângulo mede 74 cm. Quais são suas medidas, sabendo-se que o comprimento tem cinco centímetros a mais que a largura?
				\item Eu tenho R\$ 20,00 a mais que Paulo e Mario R\$ 14,00 a menos que Paulo. Nós temos juntos R\$ 156,00. Quantos reais tem cada um?
				\item A soma de dois números consecutivos é 51. Quais são esses números?
				\item A soma de dois números consecutivos é igual a 145. Quais são esses números?
				\item A soma de um número com seu sucessor é 71. Qual é esse número?
				\item A soma de três números consecutivos é igual a 54. Quais são esses números?
				\item A soma de dois números inteiros e consecutivos é - 31. Quais são esses números?
				\item A soma de dois números ímpares consecutivos é 264. Quais são esses números?	
				\item Carlos tem 17 anos e Mário tem 15 anos. Daqui a quantos anos a soma de suas idades será 72 anos?
				\item Um homem tem 25 anos de idade e seu filho 7 anos. Daqui a quantos anos a idade do pai será o triplo da idade do filho?
				\item Dois irmãos tem 32 e 8 anos respectivamente. Quantos anos faltam para que a idade do mais velho seja o triplo da idade do mais novo?
				\item Se hoje Pedro tem o dobro da idade de Maria e daqui a 20 anos Maria será 10 anos mais jovem do que Pedro, qual será a idade de Pedro nessa época?
					\begin{multicols}{5}
					\begin{enumerate}
						\item 30 anos
						\item 35 anos
						\item 40 anos  
						\item 45 anos
						\item 50 anos
					\end{enumerate}
					\end{multicols}
				\item Os estudantes de uma classe organizaram sua festa de final de ano, devendo cada um contribuir com R\$135,00 para as despesas. Como 7 alunos deixaram a escola antes da arrecadação e as despesas permaneceram as mesmas, cada um dos estudantes restantes teria de pagar R\$27,00 a mais. No entanto, o diretor, para ajudar, colaborou com R\$630,00. Quanto pagou cada aluno participante da festa?
					\begin{multicols}{5}					
					\begin{enumerate}					
						\item R\$136,00
						\item R\$138,00
						\item R\$140,00
						\item R\$142,00
						\item R\$144,00
					\end{enumerate}
					\end{multicols}
				\item Pedro e Paula são irmãos. Pedro tem 8 anos e a irmã é 2 anos mais velha que ele. Somando-se a idade dos dois e dobrando o resultado, tem-se a idade da mãe deles. Quantos anos a mãe deles tem?
				\item A soma de um número com o seu antecessor é igual a 49. Qual é o menor desses números?
				\item Uma sorveteria vendeu 900 sorvetes durante o verão. Sabendo que o valor médio dos sorvetes é de R\$ 5,00 e de que o custo médio é de R\$ 3,00, qual foi o lucro da sorveteria nesse verão?
				\item Carlos juntou a mesada de três meses para comprar um brinquedo de R\$ 60,00. Qual é o valor da mesada dele?
				\item No centro de São Paulo existe um estacionamento para carros e motos. Sabendo que o número total de rodas é 180 e que o número de carros é igual a 30, determine o número de motos.
				\item O quíntuplo de um número mais 15 é igual ao dobro desse número adicionado de 45. Qual é esse número?
				\item Beatriz passou 1/3 do dia dormindo, 1/6 na escola e 1/4 brincando com as amigas. Quantas horas restaram para ela fazer outras atividades nesse dia?
				\item Uma fazenda tem vacas e galinhas. Sabendo-se que existem 16 vacas e que o número de patas é igual a 100. Determine o número de galinhas.
				\item Numa sala de aula existem 6 meninos a mais do que meninas. Se o número total de alunos é igual a 36, determine o número de meninos.
				\item Henrique quebrou o cofrinho de moedas dele para ter dinheiro para comprar um presente. Sabendo que ele tinha R\$ 9,30, que ele não guardava moedas de 1, 5 e 25 centavos e que tinha moedas de 1 real e de 50 e 10 centavos, determine: ele poderia ter no máximo quantas moedas de 50 centavos?
				\item O dobro da quantia que Marcos possui e mais R\$ 15,00 dá para comprar exatamente um objeto que custa R\$ 60,00. Quanto Marcos possui?
				\item Um número somado com sua metade é igual a 45. Qual é esse número?
				\item (CESGRANRIO) José viaja 350 quilômetros para ir de carro de sua casa à cidade onde moram seus pais. Numa dessas viagens, após alguns quilômetros, ele parou para um cafezinho. A seguir, percorreu o triplo da quantidade de quilômetros que havia percorrido antes de parar. Quantos quilômetros ele percorreu após o café?
				\item 4.(CESPE/UnB-Adaptada) Um motorista, após ter enchido o tanque de seu veículo, gastou 1/5 da capacidade do tanque para chegar à cidade A; gastou mais 28 L para ir da cidade A até a cidade B; sobrou, no tanque, uma quantidade de combustível que corresponde a 1/3 de sua capacidade. Quando o veículo chegou à cidade B, havia, no tanque menos de:
				\begin{multicols}{5}				
				\begin{enumerate}
					\item 10 L
					\item 15 L
					\item 18 L
					\item 20 L
					\item 21 L
				\end{enumerate}
				\end{multicols}
				\item (OMSP-Adaptada) Eduardo tem R\$ 1.325,00 e Alberto, R\$ 932,00. Eduardo economiza R\$ 32,90 por mês e Alberto, R\$ 111,50. Depois de quanto tempo terão quantias iguais?
				\begin{multicols}{4}				
				\begin{enumerate}					
					\item 3 meses
					\item 5 meses
					\item 7 meses
					\item 9 meses
				\end{enumerate}	
				\end{multicols}
			\end{list}	

\section{Atividades Extras}				
\begin{list}{\textbf{Questão \arabic{quest}.}}{\usecounter{quest}}
%define a margem da lista.	
%\setlength{\labelwidth}{-2mm} \setlength{\parsep}{0mm}
%\setlength{\topsep}{0mm} \setlength{\leftmargin}{-2mm}
\renewcommand{\labelenumi}{(\alph{enumi})}
		

\item Resolva as equações
\begin{multicols}{2}				
\begin{enumerate}					
	\item $6x = 2x + 16$
	\item $2x - 5 = x + 1$
	\item $2x + 3 = x + 4$
	\item $5x + 7 = 4x + 10$
	\item $4x - 10 = 2x + 2$
	\item $4x - 7 = 8x - 2$
	\item $2x + 1 = 4x - 7$
	\item $9x + 9 + 3x = 15$
	\item $16x - 1 = 12x + 3$
	\item $3x - 2 = 4x + 9$
	\item $5x -3 + x = 2x + 9$
	\item $17x - 7x = x + 18$
	\item $x + x - 4 = 17 - 2x + 1$
	\item $x + 2x + 3 - 5x = 4x - 9$
	\item $5x + 6x - 16 = 3x + 2x - 4$
	\item $5x + 4 = 3x - 2x + 4$
\end{enumerate}	
\end{multicols}

\item Resolva as seguintes equações
\begin{multicols}{2}				
\begin{enumerate}					
	\item $4x - 1 = 3 ( x - 1)$
	\item $3( x - 2) = 2x - 4$
	\item $2( x - 1) = 3x + 4$
	\item $3(x - 1) - 7 = 15$
	\item $7 ( x - 4) = 2x - 3$
	\item $3 ( x -2) = 4(3 - x)$
	\item $3 ( 3x - 1) = 2 ( 3x + 2)$
	\item $7 ( x - 2 ) = 5 ( x + 3 )$
	\item $3 (2x - 1) = -2 ( x + 3)$
	\item $5x - 3( x +2) = 15$
	\item $2x + 3x + 9 = 8(6 -x)$
	\item $4(x+ 10) -2(x - 5) = 0$
	\item $3 (2x + 3 ) - 4 (x -1) = 3$
	\item $7 (x - 1) - 2 ( x- 5) = x - 5$
	\item $2 (3 - x ) = 3 ( x -4) + 15$
	\item $3 ( 5 - x ) - 3 ( 1 - 2x) =$
	\item $( 4x + 6) - 2x = (x - 6) + 10 +14$
	\item $( x - 3) - ( x + 2) + 2( x - 1) - 5 = 0$
	\item $3x -2 ( 4x - 3 ) = 2 - 3( x - 1)$
	\item $3( x- 1) - ( x - 3) + 5 ( x - 2) = 18$
	\item $5( x - 3 ) - 4 ( x + 2 ) = 2 + 3( 1 - 2x)$
\end{enumerate}	
\end{multicols}

\item Resolva as seguintes equações
\begin{multicols}{2}				
\begin{enumerate}					
	\item $2x + 5 - 5x = -1$
	\item $5 + 6x = 5x + 2$
	\item $x + 2x - 1 - 3 = x$
	\item $-3x + 10 = 2x + 8 +1$
	\item $5x - 5 + x = 9 + x$
	\item $7x - 4 - x = -7x + 8 - 3x$
	\item $-x -5 + 4x = -7x + 6x + 15$
	\item $3x - 2x = 3x + 2$
	\item $2 - 4x = 32 - 18x + 12$
	\item $2x - 1 = -3 + x + 4$
	\item $3x - 2 - 2x - 3 = 0$
	\item $10 - 9x + 2x = 2 - 3x$
	\item $4x - 4 - 5x = -6 + 90$
	\item $2 - 3x = -2x + 12 - 3x$
\end{enumerate}	
\end{multicols}

\item Resolva as seguintes equações
\begin{multicols}{2}				
\begin{enumerate}					
	\item $7(x - 5) = 3 (x + 1)$
	\item $3 ( x - 2 ) = 4 (-x + 3)$
	\item $2 (x +1) - (x -1) = 0$
	\item $5(x + 1) -3 (x +2) = 0$
	\item $13 + 4(2x -1) = 5 (x +2)$
	\item $4(x + 5) + 3 (x +5)= 21$
	\item $2 (x +5 ) - 3 (5 - x) =10$
	\item $8 ( x -1) = 8 -4(2x - 3)$
\end{enumerate}	
\end{multicols}


\item resolva as seguintes equações, sendo
\begin{multicols}{2}				
\begin{enumerate}					
	\item $ x /2 - x/4 = 1 /2$
	\item $x/2 - x/4 = 5$
	\item $x/5 + x/2 = 7/10$
	\item $x/5 + 1 = 2x/3$
	\item $x/2 + x/3 = 1$
	\item $x/3 + 4 = 2x$
	\item $x/2 + 4 = 1/3$
	\item $5x/3 - 2/5 = 0$
	\item $x - 1 = 5 - x/4$
	\item $x + x/2 = 1$
	\item $8x/3 = 2x - 9$
	\item $x/2 + 3/4 = 1/6$
\end{enumerate}	
\end{multicols}

\item Resolva as seguintes equações
\begin{multicols}{2}				
\begin{enumerate}					
	\item $x/2 - 7 = x/4 + 5$
	\item $2x - 1/2 = 5x + 1/3$
	\item $x - 1 = 5 - x/4$
	\item $x/6 + x/3 = 18 - x/4$
	\item $x/4 + x/6 + x/6 = 28$
	\item $x/8 + x/5 = 17 - x/10$
	\item $x/4 - x/3 = 2x - 50$
	\item $5x /2 + 7 = 2x + 4$
	\item $x/4 - x/6 = 3$
	\item $3x/4 - x/6 = 5$
	\item $x/5 + x/2 = 7/10$
	\item $(2x - 7)/5 = (x + 2)/3$
	\item $5x/2 = 2x + (x - 2) / 3$
	\item $(x - 3)/4 - (2x - 1) / 5 = 5$
\end{enumerate}	
\end{multicols}

\item Resolva as seguintes equações
\begin{multicols}{2}				
\begin{enumerate}					
	\item $ x/2 + x/3 = (x + 7)/3$
	\item $(x + 2) / 6 + (x +1)/4 = 6$
	\item $(x -2) /3 - (x + 1)/ 4 =4$
	\item $(x - 1) /2 + (x - 2) /3 = (x -3)/4$
	\item $(2x- 3) / 4 - (2 - x)/3 = (x -1) / 3$
	\item $(3x -2) / 4 = (3x + 3) / 8$
	\item $ 3x + 5) / 4 - (2x - 3) / 3 = 3$
	\item $ x/5 - 1 = 9$
	\item $x/3 - 5 = 0$
	\item $x/2 + 3x/5=6$
	\item $5x - 10 = (x+1)/2$
	\item $(8x - 1) / 2 - 2x = 3$
	\item $(x - 1) /2 + (x - 3)/3 = 6$
	\item $(5x - 7)/2 = 1/2 + x$
	\item $(2x - 1) / 3 = x - (x - 1)/5$
\end{enumerate}	
\end{multicols}

\item A soma do quádruplo de um número com 63 é igual a 211. Qual é esse número?

\item O sêxtuplo de um número, diminuído de 12, é igual a 36. Qual é esse número?

\item O quíntuplo de um número, aumentado de 100, é igual a 300. Qual é esse número?

\item O triplo de um número menos 99 é igual a 9. Qual é esse número?

\item A soma de um número com o seu sucessor é 37. Qual é esse número?

\item A soma de dois números consecutivos é igual a 101. Quais são esses números?

\item Somando 20 anos ao quíntuplo da idade de Arthur, obtemos 40 anos. Qual é a idade de Arthur?

\item Pensei em um número que multiplicado por 8 e subtraído 16 dá 64. Qual é esse número?

\item Pensei em um número que somado com seu dobro e diminuído de 5 é igual a 37. Qual é esse número?

\item O dobro de um número mais 10 é igual a 56. Qual é esse número?

\item Júnior e Luís jogam na mesma equipe de basquete. No último jogo dessa equipe, os dois juntos marcaram 52 pontos. Júnior marcou 10 pontos a mais que Luís. Quantos pontos Júnior marcou nessa partida?

\item O dobro da quantia que Jair possui e mais 18 reais dá 60 reais. Quantos reais Jair possui?

\item Uma fita de 247 metros vai ser dividida em duas partes, de modo que uma tenha 37 metros a mais que a outra. Quanto mede a parte maior?

\item O sêxtuplo de um número menos 10 é igual ao dobro desse próprio número mais 14. Qual é esse número?

\item Pensei em um número que multiplicado por 9 e subtraído 81 dá 18. Qual é esse número?

\item A soma da minha idade, com a idade de meu irmão que é 7 anos mais velho que eu dá 37 anos. Quantos anos eu tenho de idade?

\item O quíntuplo de um número menos 15 é igual ao dobro desse mesmo número mais 12. Qual é esse número?

\item O quádruplo de um número menos 20 é igual ao triplo desse mesmo número mais 37. Qual é esse número?

\item O triplo de um número mais 30 é igual a esse próprio número mais 70. Qual é esse número?

\item O triplo de um número mais dois é igual ao próprio número mais 8. Qual é esse número?

\item O quádruplo de um número, diminuído de três, é igual a 33. Qual é esse número?

\item O quíntuplo de um número mais 20 é igual ao próprio número mais 16. Qual é esse número?

\item O quádruplo de um número mais 10 é igual ao dobro desse mesmo número menos 4. Qual é esse número?

\item O triplo de um número menos 10 é igual ao dobro desse mesmo número menos 78. Qual é esse número?

\item O dobro de um número mais 50 é igual ao próprio número menos 2. Qual é esse número?

\item A terça de um número menos 3 é igual a 9. Qual é esse número?

\item A metade de um número mais 20 é igual a 25. Qual é esse número?

\item A sexta parte de um número menos 4 é igual a 2. Qual é esse número?

\item A quarta parte de um número mais 30 é igual a 40. Qual é esse número?

\item A metade de um número somada com sua terça parte é igual a 25. Qual é esse número?
\end{list}