\chapter{Progressões Aritméticas}
	\section{Atividada a serem reorganizadas}
	\begin{list}{\textbf{Questão \arabic{quest}.}}{\usecounter{quest}}
%define a margem da lista.	
%\setlength{\labelwidth}{-2mm} \setlength{\parsep}{0mm}
%\setlength{\topsep}{0mm} \setlength{\leftmargin}{-2mm}
\renewcommand{\labelenumi}{(\alph{enumi})}

		\item Determine o 52º termo da PA ( 8, 14, 20, 26, ...) 
		\begin{multicols}{4}
		\begin{enumerate}
			\item 308
			\item 326
			\item 314
			\item 320
		\end{enumerate} 
		\end{multicols}

		\item Determine o 43 termo da PA (-30, -25, -20, -15, ...) 
		\begin{multicols}{4}
		\begin{enumerate}
			\item 240
			\item 180
			\item 200
			\item 195
		\end{enumerate} 
		\end{multicols}

		\item Encontre a razão da PA, tal que $a_{1} = 15$ e $a_{16} = 45$:
		 \begin{multicols}{4}
		\begin{enumerate}
			\item 1
			\item 0
			\item -1 
			\item 2 
		\end{enumerate} 
		\end{multicols}
		
		\item Encontre a razão da PA, tal que $a_{1} = -5$ e $a_{301} = 1195$: 
		\begin{multicols}{4}
		\begin{enumerate}
			\item 395
			\item 400
			\item 380 
			\item  250
		\end{enumerate} 
		\end{multicols}
		
		\item O primeiro termo de uma PA é 100 e o trigésimo é 187. Qual a soma dos trinta primeiros termos? 
		\begin{multicols}{4}
		\begin{enumerate}
			\item 287
			\item 5650
			\item 4305 
			\item 2355   
		\end{enumerate} 
		\end{multicols}
		
		\item Sabendo que o primeiro termo de uma PA vale 21 e a razão é 7, calcule a soma dos 12 primeiros termos desta PA: 
		\begin{multicols}{4}
		\begin{enumerate}
			\item 714
			\item 650
			\item 305 
			\item 355 
		\end{enumerate} 
		\end{multicols}
		
		\item O sétimo termo de uma PA é 20 e o décimo é 32. Então o vigésimo termo é 
		\begin{multicols}{4}
		\begin{enumerate}
			\item 60
			\item 59
			\item 72 
			\item 76 
		\end{enumerate} 
		\end{multicols}
		
		\item O número mensal de passagens de uma determinada empresa aérea aumentou no ano passado nas seguintes condições: em janeiro foram vendidas 33.000 passagens; em fevereiro, 34.500; em março, 36.000. Esse padrão de crescimento se mantém para os meses subsequentes. Quantas passagens foram vendidas por essa empresa em julho do ano passado?
		\begin{multicols}{5}
		\begin{enumerate}
			\item 38.000
			\item 40.500
			\item 41.000 
			\item 42.000
			\item 48.000 
		\end{enumerate} 
		\end{multicols}

		\item Numa PA em que $a_{1} = 2$ e $a_{20} = 10$ Qual é a soma dos 20 primeiros termos dessa PA?
		\begin{multicols}{5}
		\begin{enumerate}
			\item 420
			\item 240
			\item 300 
			\item 300 
			\item 120.
		\end{enumerate} 
		\end{multicols}

		\item Numa PA em que $a_{6} = 2$ e $a_{38} = 10$ Qual é a soma dos 20 primeiros termos dessa PA?
		\begin{multicols}{5}
		\begin{enumerate}
			\item 120/3
			\item 125/4
			\item 150/3 
			\item 125/2 
			\item 100.
		\end{enumerate} 
		\end{multicols}

		\item Qual é a soma dos 20 primeiros pares não negativos?
		\begin{multicols}{5}
		\begin{enumerate}
			\item 420
			\item 400
			\item 380 
			\item 300 
			\item 100.
		\end{enumerate} 
		\end{multicols}

		\item Qual é a soma dos 20 primeiros múltiplos positivos de 7?
		\begin{multicols}{5}
		\begin{enumerate}
			\item 7000
			\item 147
			\item 1407 
			\item 7007 
			\item 140.
		\end{enumerate} 
		\end{multicols}
		\input{fim_da_lista}
