\chapter{Sistemas de Equações do Primeiro Grau.}

\section{Atividades Basicas}
	\begin{list}{\textbf{Questão \arabic{quest}.}}{\usecounter{quest}}
%define a margem da lista.	
%\setlength{\labelwidth}{-2mm} \setlength{\parsep}{0mm}
%\setlength{\topsep}{0mm} \setlength{\leftmargin}{-2mm}
\renewcommand{\labelenumi}{(\alph{enumi})}

	
	\item Utilizando o método da substituição, determine a solução de cada um dos seguintes sistemas de equações do primeiro grau nas incógnitas $x$ e $y$.
	\begin{multicols}{2}	
	\begin{enumerate}
		\item $\begin{cases}	x+y=20\\ x-y=8\end{cases}$
		\item $\begin{cases}	3x-y=18\\x+y=10\end{cases}$
		\item $\begin{cases} 2x+y=-3\\ x-3y=-26\end{cases}$
		\item $\begin{cases}	x+5y=-24\\ 3x-2y=-4\end{cases}$
		\item $\begin{cases}	\displaystyle\frac{x}{5}=10+\frac{y}{2}\\ x-y=29\end{cases}$
		\item $\begin{cases} 3x-5y=2(x-y)+1\\ 3y-3(x-3y)+x=-2-3y\end{cases}$
		\item $\begin{cases}	\displaystyle\frac{x+y}{5}=\frac{x-y}{3}\\ \displaystyle\frac{x}{2}=y+2  \end{cases}$
		\item $\begin{cases}	x=2(y+2)\\ \displaystyle\frac{x-y}{10}=\frac{x}{2}+2\end{cases}$
	\end{enumerate}
	\end{multicols}
	
	\item Determine a solução de cada um dos seguintes sistemas de equações pelo método da substituição:
	\begin{multicols}{2}	
	\begin{enumerate}
		\item $\begin{cases}5x+y=-1\\ 3x+4y=13\end{cases}$
		\item $\begin{cases}	x+\displaystyle\frac{y}{2}=12\\ \displaystyle\frac{x+y}{2}+\frac{x-y}{3}=10\end{cases}$
		\item $\begin{cases}2(x-3)+y=-15\\ \displaystyle\frac{x}{4}=\frac{x+y}{6}+\frac{2}{3}\end{cases}$
		\item $\begin{cases}2x-5=y-4\\ 7x-y=y+3\end{cases}$	
	\end{enumerate}
	\end{multicols}
	
	\item Resolva em seu caderno esses sistemas pelo método da adição.
	\begin{multicols}{2}
	\begin{enumerate}
		\item $\begin{cases} 3x-2y=10\\ 5x+2y=22 \end{cases}$
		\item $\begin{cases}-a+2b=7\\ a-3b=-9\end{cases}$
		\item $\begin{cases}a+3b=5\\ 2a-3b=-8\end{cases}$
		\item $\begin{cases}2a-b=-3\\ 6a+b=7\end{cases}$
	\end{enumerate}
	\end{multicols}
	
	\item Utilizando o método da adição , determine a solução de cada um dos seguintes sistemas de equações do primeiro grau nas incógnitas $x$ e $y$.
	\begin{multicols}{2}	
	\begin{enumerate}
		\item $\begin{cases} x+y=32\\ x-y=18\end{cases}$
		\item $\begin{cases} 6x-3y=20\\ 4x+3y=40\end{cases}$
		\item $\begin{cases}	7x+6y=23\\ 5x+6y=21\end{cases}$
		\item $\begin{cases} 8x+5y=11\\ 4x+5y=3\end{cases}$
		\item $\begin{cases}	2x-3y=11\\ 2x+7y=1\end{cases}$
		\item $\begin{cases} 2x-y=12\\ \displaystyle\frac{x}{3}+ \frac{y}{2}=6\end{cases}$
		\item $\begin{cases}	3(x-2)=2(y-3)\\ 18(y-2)+y=3(2x+3)\end{cases}$
		\item $\begin{cases}	\displaystyle\frac{x-y}{5}=\frac{x-y}{2}\\ \displaystyle\frac{3x}{2}=y-2 \end{cases}$
	\end{enumerate}
	\end{multicols}
	
	\item Usando o método mais conveniente, determine a solução de cada um dos seguintes sistemas de equações.
	\begin{multicols}{3}
	\begin{enumerate}
		\item $\begin{cases}	3x-20=y-4\\ \displaystyle\frac{x+1}{3}=\frac{y+2}{2}+\frac{x}{6} \end{cases}$
		\item $\begin{cases}\displaystyle\frac{5x-2}{2}+\frac{y-3}{5}=2x\\ \displaystyle\frac{7(y-1)}{2}+\frac{y-3}{3}=2y\end{cases}$
		\item $\begin{cases} \displaystyle\frac{x-y}{6}+\frac{x+y}{8}=5\\ \displaystyle\frac{x+y}{4}-\frac{x-y}{5}=10\end{cases}$
	\end{enumerate}
	\end{multicols}
	
	\item Calcule o valor de número real representado pela letra $z$ na igualdade de $x-2y+z=-12$, sabendo que :$$\begin{cases} x+y=18\\ 2x-y=12 \end{cases}$$
				
	\item o par ordenado $(x,y)$ é a solução do sistema:$$\begin{cases} \displaystyle\frac{x+5}{5}=y-\displaystyle\frac{y}{2}\\ \displaystyle\frac{5x}{2}+3(y-10)=5(x-10) \end{cases}$$Nessas condições, determine o valor de :
	\begin{multicols}{3}	
	\begin{enumerate}
	\item $xy$ 
	\item $x^2+y^2$
	\item $\displaystyle\frac{x}{y}$
	\end{enumerate}
	\end{multicols}
	
	\item É dado o sistema $$\begin{cases} \displaystyle{2(x+2)-3(y-2)=5,6}\\ \displaystyle\frac{x}{2}+\frac{y}{4}=0,45 \end{cases}$$ Sendo $(x,y)$ a solução do sistema calcule $y^3-x^3$.
	
	\item Sabendo que $x+y=15$, $x+z=11$ e $y+z=6$, determine o valor da soma $x+y+z$.
	\end{list}
\section{Problemas Envolvendo Sistemas de Equações}
	\begin{list}{\textbf{Questão \arabic{quest}.}}{\usecounter{quest}}
%define a margem da lista.	
%\setlength{\labelwidth}{-2mm} \setlength{\parsep}{0mm}
%\setlength{\topsep}{0mm} \setlength{\leftmargin}{-2mm}
\renewcommand{\labelenumi}{(\alph{enumi})}

	\item Resolva os problemas a seguir:
	\begin{enumerate}
		\item A soma de dois números é 32 e a diferença é 8. Quais são esses números?
		\item A soma de dois números é igual a 27 e a diferença é 7. Quais são esses números?
		\item A soma de dois números é igual a 37 e a diferença é 13. Quais são esses números?			
		\item A diferença entre dois números reais é 7. Sabe-se também que a soma do dobro do primeiro número com o quádruplo do segundo é 11. Quais são esses números?
		
		\item Josias comprou 5 canetas e 3 lápis e gastou R\$21,10. Mariana comprou 3 canetas e 2 lápis e gastou R\$ 12,90. Fernando comprou 2 canetas e 5 lápis. Quanto ele gastou?
		
		\item Em uma sala de aula retangular, o primeiro é de 44 m e a diferença entre a metade de mediada do comprimento e a quarta parte da medida da largura é 5 m. Descubra a área dessa sala de aula.
		\item A soma de dois números é 127 e a diferença entre eles é 49. Quais são esses números?
	\end{enumerate}
	\item Num estacionamento há carros e motos, totalizando 78 veículos. O número de carros é o quíntuplo do número de motos. Quantas motos existem neste estacionamento?
	
	\item Um senhor tem coelhos e galinhas num total de 20 cabeças e 58 pés. Determine o número de coelhos e de galinhas.	
	
	\item Eu tenho 30 cédulas, algumas de R\$ 5,00 e outras de R\$ 10,00. O valor total das cédulas é de R\$ 250,00. Quantas cédulas de R\$ 5,00 e quantas cédulas de R\$ 10,00 eu tenho?
	
	\item Num pátio há bicicletas e carros num total de 20 veículos e 56 rodas. Determine o número de bicicletas e de carros.
	
	\item A soma de dois números é $169$, e a diferença entre eles é $31$. Quais são os dois números?
	
	\item A soma de dois números é $110$. O maior deles é igual ao triplo do menor mais $18$ unidades. Quais são os dois números?
	
	\item Um terreno retangular tem $128$ m de perímetro. O comprimento tem $20$ m a mais que a largura. Determine as dimensões desse terreno e a sua área.
	
	\item Pelo regulamento de um torneio de basquete, cada partida que a equipe ganha vale 2 pontos e cada partida que perde vale $1$ ponto. A equipe de basquete do nosso colégio, disputando esse torneio, jogou $10$ vezes e já acumulou $16$ pontos. Quantos jogos a equipe do nosso colégio já venceu?
	
	\item Uma tábua de $235$ cm de comprimento foi dividida em $3$ partes. A primeira tem $85$ cm de comprimento, e a segunda tem o dobro do comprimento da terceira parte. Quais são os comprimentos dessas duas últimas partes?
	
	\item Um treinador propôs a um de seus jogadores que a arremessasse, sucessivamente, uma bola à cesta, informando que ele ganharia $5$ pontos a cada acerto e perderia $2$ pontos a cada erro. Ao fim dessa parte do treinamento, o jogador havia feito $36$ arremessos e conseguido acumular $110$ pontos. Determine o número de arremessos certos.
	
	\item Num terreno há galinhas e ovelhas, num total de $21$ animais e $50$ pés. Quantos animais de cada espécie há nesse terreiro?
	
	\item Neste ano, os $300$ alunos do 8$^{\underline{o}}$ ano de uma escola estão divididos nas aulas de Inglês em Nível Avançado e Nível Intermediário. Para o próximo ano a escola resolveu oferecer o curso de Espanhol como opção ao curso de Inglês. Sabendo que $40\%\ (0,4)$ dos alunos do Nível Avançado e $5\%\ (0,05)$ dos alunos do Nível Intermediário se inscreveram no novo curso, totalizando $50$ alunos, quantos alunos do Nível Intermediário de Inglês não escolheram o curso de Espanhol?
	
	\item Duas pessoas tem juntas, $70$ anos. Subtraindo $10$ anos da idade da mais velha e acrescentando os mesmos $10$ anos à idade da mais jovem, as idades ficam iguais. Qual é a idade de cada pessoa?
	
	\item Um campeonato de Fórmula 1 termina com o campeão levando 7 pontos de vantagem sobre o vice-campeão. O campeão e o vice, juntos, somam no final da temporada 173 pontos. Nessas condições, quantos pontos somou o campeão da temporada? E o vice?
	
	\item Para embalar $1650$ livros, uma editora utilizou $27$ caixas, umas com capacidade para $50$ livros e outras, para $70$ livros. Quantas caixas de cada tipo a editora utilizou?
	
	\item Um marceneiro é contratado para colocar prateleiras em uma parede de um depósito que tem 6 m de altura. O dono do depósito quer que sejam colocadas 23 prateleiras, com alguns vãos de 20 cm e outros de 30 cm. Nessas condições, quantos vãos de 20 cm e quantos vãos de 30 cm o marceneiro vai deixar? 
	
	\item Certa mercadoria é vendida nas lojas A e B. O preço dessa mercadoria é de 18 reais mais caro na loja A. Se a loja A oferecer um desconto de $20\%$, o preço nas duas lojas será o mesmo. Qual é o preço inicial da mercadoria em cada uma das lojas?
	
	\item Uma herança de R\$ $50.000,00$ foi deixada para dois irmãos. No testamento, ficou estabelecido que o filho mais novo deveria receber R\$ $18.000,00$ a mais do que o irmão mais velho. Qual a parte que cabe a cada um?
	
	\item Leia o que afirmaram Cibele, Mariana, e Gustavo sobre compras de cadernos e canetas em uma papelaria:
	\begin{itemize}
		\item Cibele: Comprei dois cadernos e uma caneta e paguei R\$ 14,00.
		\item Mariana:Eu comprei um caderno e duas canetas e paguei R\$ 10,00.
		\item Gustavo: Então, cada caderno custa R\$ 5,00 e cada caneta, R\$ 4,00.
	\end{itemize}
	\begin{enumerate}
		\item Escreva um sistema correspondente às duas primeiras afirmações.
		\item Será que a afirmação de Gustavo está correta?
		\item Descubra o preço da cada caderno e de cada caneta.
	\end{enumerate}
	
	\item O ``peso'' de Camila e de seu gato Tico, juntos, é de 32 kg, O ``peso'' de Camila é 7 vezes o de Tico. Qual o ``peso'' de cada um?
	
	\item Ana e Marcelo economizaram suas mesadas para comprar um presente para o pai deles. Juntando a quantia dos dois, dá para comprar um tênis que custa R\$ 55,00 e não sobra troco. A quantia que Ana tem ultrapassa em R\$ 21,00 a quantia de Marcelo. Quantos reais tem cada um?
	\end{list}