\chapter{Problemas para o 8º ano}
\section{Problemas extras sobre as quatro operações.}
\begin{list}{\textbf{Questão \arabic{quest}.}}{\usecounter{quest}}
%define a margem da lista.	
%\setlength{\labelwidth}{-2mm} \setlength{\parsep}{0mm}
%\setlength{\topsep}{0mm} \setlength{\leftmargin}{-2mm}
\renewcommand{\labelenumi}{(\alph{enumi})}
	
    \item Fabrício tinha 320 reais para pagar as contas (117 reais de energia elétrica, 58 reais de água e 88 reais de telefone) e para fazer algumas compras. Quanto lhe restou para fazer as compras?
     
    \item Na escola de Pedro há 8 classes de 35 alunos, 5 classes de 33 alunos e 12 classes de 30 alunos. Qual é o total de alunos nessa escola?
     
    \item Quarenta e cinco balas foram repartidas entre 3 crianças, Ana, Maria e João. Quantas balas cada uma recebeu?
     
    \item Ana tinha 500 reais no banco. Na segunda-feira retirou 250 reais e na terça-feira fez um depósito de 180 reais. Qual o valor do seu saldo?
     
    \item Fernando comprou 4 cadernos e pagou R\$ 15,00. Quanto pagaria se tivesse comprado 12 cadernos?
     
    \item O preço de três camisetas é R\$ 54,00. Noemi deu uma nota de R\$ 50,00 para pagar a compra de duas camisetas. Quanto recebeu de troco?
     
    \item Aline tinha uma certa quantia na bolsa. Emprestou R\$ 15,00 ao seu irmão e agora tem R\$65,00. Qual quantia ela tinha inicialmente na bolsa?
     
    \item Laura e Célia jogaram uma partida de videogame em 5 rodadas. Laura fez nas três primeiras rodadas 47 pontos em cada uma e, nas seguintes, 51 e 49 pontos. Célia fez 50 pontos na primeira rodada e, nas seguintes, 48 pontos em cada uma. Responda às perguntas a seguir.
	\begin{enumerate}[a)]
    	\item Quem fez mais pontos na partida?
    	\item Quantos pontos faltaram para Laura totalizar 270 pontos?
	\end{enumerate}      
	
	\item Na escola de Pedro há 7 classes de 35 alunos, 9 classes de 33 alunos e 12 classes de 30 alunos. Qual é o total de alunos nessa escola?
	
	\item Se 65 reais é o preço de 13 cadernos, quanto pagaremos por 20 cadernos?
	
	\item Para plantar 532 mudas de rosas em 14 canteiros, com a mesma quantidade de mudas em todos eles, quantas mudas Lauro precisa colocar em cada canteiro?
	
	\item . Uma farmácia possui na prateleira um analgésico com 8 comprimidos em cada cartela. Cada caixa desse analgésico contém 25 cartelas. Na prateleira, estão 3 caixas fechadas e 1 caixa com 12 cartelas. Qual é o total de comprimidos desse analgésico?
	
	\item  Uma loja de meias possui em seu estoque 20 caixas de meias pretas, 15 caixas de meias brancas e 14 caixas de meias azuis. As caixas com meias pretas e brancas contêm 36 pares em cada caixa e as restantes, 18 pares em cada caixa. Qual é o total de meias do estoque? Quantos pares fazem parte do estoque?
	
	\item Uma fábrica de massas produz massas de pizza de dois tipos: pequena e grande. Os pacotes com pizzas pequenas contêm 10 massas cada um e os pacotes com pizzas grandes contêm 2 massas cada um. Um carregamento saiu da fábrica com 4700 pizzas entre pequenas e grandes. Sabe-se que 1180 pizzas são grandes. Quantos pacotes de pizzas pequenas e grandes estão no carregamento?
	
	\item Um conjunto habitacional possui 16 prédios, sendo 9 prédios de 6 andares cada um, com quatro apartamentos por andar, e os restantes com 5 andares cada um, com 6 apartamentos por andar. Qual é o total de apartamentos desse conjunto habitacional?
	
	\item Meu livro tem 160 páginas. Já li 92. quero terminar a leitura em 4 dias lendo o mesmo número de páginas em cada dia. Quantas páginas lerei por dia?
	
	\item Três garçons juntaram as gorjetas do dia: 70 notas de 10 reais e 22 notas de 5 reais. Dividindo igualmente esse total, quanto receberá cada um?
	
	\item Um caminhão transporta 85 sacas de café em cada viagem. Quantas viagens ele terá de fazer para transportar 680 sacas de café?
	
	\item Para uma pessoa estar bem alimentada, é recomendável que ela tome 2 copos de leite por dia. Quantos copos de leite uma pessoa deve tomar em um ano?
	
	\item Para uma pessoa estar bem alimentada, é recomendável que ela tome 2 copos de leite por dia. Quantos copos de leite uma pessoa deve tomar em um ano?
	
	\item Um prédio comercial tem 20 andares. Até o 8º andar existem 24 escritórios por andar. Nos demais, existem 13 escritórios por andar. Quantos escritórios existem até o 8º andar. Quantos escritórios existem no prédio?
	
	\item Júlia e sua prima compraram 108 carrinhos e pagaram R\$ 216,00. Deste valor sua prima pagou R\$ 126,00 pelos seus carrinhos.
	\begin{multicols}{2}
	\begin{enumerate}[a)]
		\item Qual foi o valor pago por Júlia?
		\item Quanto custa cada carrinho?
		\item Quantos carrinhos Júlia comprou?
		\item  Quantos carrinhos a prima comprou?
	\end{enumerate}
	\end{multicols}
		
	\item Um comerciante comprou por R\$ 720,00 um atum com 20 kg. Ele vendeu cada quilo de atum por R\$ 60,00.
	\begin{enumerate}[a)]
		\item Quanto o comerciante pagou por cada quilo de atum?
		\item Quanto ele faturou ao vender os 20 kg de atum?
		\item Qual foi o lucro total do comerciante ao vender os 20 kg de atum? (lembrando que lucro é o faturamento menos o valor pago pelo comerciante)
	\end{enumerate}	

	\item Dona Márcia comprou 7 dúzias de bananas. Distribuiu duas para cada macaco e ela reservou 12 para levar para casa. Quantos macacos ela alimentou?

	\item  Sílvia comprou uma geladeira por R\$ 820,00. Ela deu R\$ 220,00 de entrada e pagou o restante em três prestações mensais de igual valor. Qual o valor de cada prestação?

	\item Na decisão das Olimpíadas, foram realizadas três partidas de basquete. Na primeira partida, compareceram 2.853 pessoas. Na segunda,1.987 e, na final, 3.587 pessoas.
	\begin{enumerate}[a)]
		\item Em qual partida o público foi maior?
		\item Qual a diferença de público entre o terceiro e o primeiro jogo?
		\item Nos três jogos, quantas pessoas compareceram no total?
	\end{enumerate}

	\item Minha mãe nasceu em 1967. Eu nasci 24 anos depois. Quantos anos eu tenho agora?
	
	\item Na banca de revistas, Mariana comprou um álbum de figurinhas que custou R\$ 1,50; um pacote de figurinhas por R\$ 0,25; uma revista em quadrinhos por R\$ 2,50 e uma revista sobre informática que custou R\$4,25.De acordo com os preços, responda:
	\begin{enumerate}[a)]
		\item Quanto Mariana gastou ao todo?
		\item Se Mariana pagar a conta com uma nota de R\$ 50, quanto receberá de troco?
	\end{enumerate}
	
	\item Um camelô comprou 30 ursinhos de pelúcia por R\$ 165,00. Desejando lucrar R\$ 75,00 com a venda desses ursinhos, por quanto o camelô deve vender cada um?

	\item Um escritor escreveu, em certo dia, as 20 primeiras páginas de um livro. A partir desse dia, ele escreveu a cada dia tantas páginas quanto havia escrito no dia anterior mais 5 páginas. Se o escritor trabalhou 4 dias, quantas páginas ele escreveu no total?

	\item  A Lotação de um Teatro é de 360 lugares, todos do mesmo preço. Uma parte da lotação foi vendida por R\$ 3.000,00, tendo ficado ainda por vender ingressos no valor de R\$ 6.000,00. Quantos ingressos já foram vendidos?

	\item Um pai tem 35 anos e seus filhos 6, 7 e 9 anos. Daqui a 8 anos, a soma das idades dos três filhos menos a idade do pai será de quanto?

	\item Em uma sala de aula, onde todos os lugares se encontram ocupados, os alunos estão sentados em filas e essas filas têm todas o mesmo número de lugares.
	
	O aluno roberto tem:
	\begin{enumerate}
		\item[$-$]um aluno sentado à sua frente;
		\item[$-$]dois alunos sentados atrás de si;
		\item[$-$]três alunos sentados à sua direita;
		\item[$-$]dois alunos sentadas à sua esquerda.
	\end{enumerate}
	Quantos alunos há na sala de Roberto?	

	\item Dispomos de cinco cadeados e 5 chaves para os mesmos. Qual o número máximo de tentativas que devemos fazer para estabelecer a correspondência correta entre os cadeados e as chaves?
\input{fim_da_lista}