\chapter{Cálculo Numérico}
\section{Operações com números reais.}
\begin{list}{\textbf{Questão \arabic{quest}.}}{\usecounter{quest}}
%define a margem da lista.	
%\setlength{\labelwidth}{-2mm} \setlength{\parsep}{0mm}
%\setlength{\topsep}{0mm} \setlength{\leftmargin}{-2mm}
\renewcommand{\labelenumi}{(\alph{enumi})}

\item Calcular o valor numérico de cada expressão.
\begin{multicols}{2}
\begin{enumerate}

	\item $\displaystyle{
			\frac{\displaystyle{\frac{4}{5}-\frac{1}{2}}}{\displaystyle{\left(\frac{3}{2}\right)^{-2}}}
			}$
	\item $0,2-\left\{(0,5)^2\cdot(-2)+\left[3:(4^0\cdot\sqrt{36})\right]-1\right\}$
\end{enumerate}
\end{multicols}
	\item Dar o valor da expressão 
		$\displaystyle{
		\frac{10^{-2}+\sqrt{400}-(8,1:3)^{3}}
		{\displaystyle\sqrt[10]{1024}-\left(\displaystyle\frac{13}{4}\right)^0}
		}$.
	\item Racionalizar o denominador da fração $\displaystyle{\frac{\sqrt{2}+2}{\sqrt{2}-2}}$.
	\item Calcular o valor da raiz quadrada de $82,81$.
	\item Calcular $12\%$ de $75\%$ de 36.

	\item Calcule o valor de cada expressão.
	%\begin{multicols}{2}
	\begin{enumerate}
		\item $\displaystyle{
			1,8 + \left\{ 13 - (5^2-3):\left[3-(0,9)^0\right]\right\}+\sqrt{0,36}
			}$
		\item $\displaystyle{
			\frac{\displaystyle\frac{1}{3}+\frac{4}{5}}{\displaystyle\frac{2}{9}\cdot\frac{3}{2}}		
		}$
		\item $\displaystyle{
			\frac{\displaystyle\frac{2}{3}\cdot\frac{1}{7}}{\displaystyle\frac{1}{5}\cdot\left(\frac{3}{10}\right)^{-1}}		
		}$
		\item $\left[3,2\cdot(0,8)^{-2}+5\right]:0,2+\left[(-5,2)^0:0,3\right]^{-1}$
		\item $\displaystyle{
			\left(\frac{1}{3}\right)^{2} + \left(\frac{3}{4}\right)^{-2} \cdot \left(\frac{2}{5}\right)^{-1}	
		}$
		\item $\frac{\displaystyle\frac{5}{3}\cdot(2,1)}{\displaystyle(6,4):\displaystyle\left(\frac{6}{4}\right)}$
		\item $\displaystyle\frac{\displaystyle\frac{2}{3}+ (0,3)^{-1}}{2,7\cdot\displaystyle\frac{1}{9}} : \frac{5:0,2}{3-\displaystyle\frac{1}{2}}$
	\end{enumerate}
	%\end{multicols}
	\item Calcule o valor das expressões.
	\begin{enumerate}
		\item $\displaystyle\frac{\sqrt{169}-(0,2:2)+10^0} {\left( \displaystyle\frac{2}{5} \cdot \frac{1}{6}\right)-\sqrt[5]{25 \cdot 125}}$ 
		\item $\displaystyle\frac{(-2,1+3\cdot 1,2) + (0,05:0,1)^{-1}}{\left[(-2)^2:(-2)^3\right]^{-1}}+\sqrt{196}$
	\end{enumerate}
	\item Dividir um número por 0,0625 é equivalente a multiplicá-lo por:
	\begin{multicols}{5}
	\begin{enumerate}
		\item 4
		\item 8
		\item 16
		\item 2
		\item 1
	\end{enumerate}
	\end{multicols}
	\item Racionalize o denominador das expressões.
	\begin{multicols}{2}
	\begin{enumerate}
		\item $\displaystyle\frac{2}{\sqrt{3}}$
		\item $\displaystyle\frac{2}{\sqrt{7}-\sqrt{5}}$
		\item $\displaystyle\frac{3}{2+\sqrt{3}}$
		\item $\displaystyle\frac{\sqrt{11} + \sqrt{7}}{\sqrt{11} - \sqrt{7}}$
		\item $\displaystyle\frac{6}{\sqrt[3]{3}}$
		\item $\displaystyle\frac{10}{\sqrt[4]{8}}$
	\end{enumerate}
	\end{multicols}
	\item Calcule os valor das expressões.
	\begin{multicols}{2}
	\begin{enumerate}
		\item $\sqrt{289}+\sqrt{441}$
		\item $\sqrt[4]{1296}-\sqrt[5]{243}$
		\item $\sqrt{72}+\sqrt{50}$
		\item $\displaystyle\frac{7}{\sqrt[3]{729}} + \frac{3}{\sqrt[7]{128}}$
		\item $\displaystyle\sqrt[4]{32} + \sqrt[3]{40} - \sqrt[4]{162}$
	\end{enumerate}
	\end{multicols}
	\item Calcule:
	\begin{multicols}{2}
	\begin{enumerate}
		\item 20\% de 450.
		\item 35\% de 800
		\item 1\% de 10\% de 9
		\item $(25\%)^2$ de 100
	\end{enumerate}
	\end{multicols}
\input{fim_da_lista}