\chapter{Progressões Geométricas.}
\begin{list}{\textbf{Questão \arabic{quest}.}}{\usecounter{quest}}
%define a margem da lista.	
%\setlength{\labelwidth}{-2mm} \setlength{\parsep}{0mm}
%\setlength{\topsep}{0mm} \setlength{\leftmargin}{-2mm}
\renewcommand{\labelenumi}{(\alph{enumi})}

\item A sequência seguinte é uma progressão geométrica, observe: (2, 6, 18, 54...). Determine o 8º termo dessa progressão

\item Um carro, cujo preço à vista é R\$ 24 000,00, pode ser adquirido dando-se uma entrada e o restante em 5 parcelas que se encontram em progressão geométrica. Um cliente que optou por esse plano, ao pagar a entrada, foi informado que a segunda parcela seria de R\$ 4 000,00 e a quarta parcela de R\$ 1 000,00. Quanto esse cliente pagou de entrada na aquisição desse carro?

\item Sabendo que uma PG tem $a_1 = 4$ e razão $q = 2$, determine a soma dos 10 primeiros termos dessa progressão. 

\item Calcule o quarto e o sétimo termos da P. G. $(3, -6, 12,...)$.

\item Determine o segundo termo de uma P. G. crescente tal que $a_1 = 8$ e $a_3 = 18$.

\item As medidas do lado, do perímetro e da área de um quadrado estão em progressão geométrica, nessa ordem. Qual é área do quadrado.

\item Determine, de modo que a sequência $(4, 4x , 10x +6)$ seja PG.

\item Determine o 15º termo da PG $(256,128,64,32,...)$

\item Determine o 10º termo da PG $(3,6,12...)$.

\item Determine a PG de três termos sabendo do que o produto desses termos é 8 a soma do 2º com o 3º termo é 10.

\end{list}