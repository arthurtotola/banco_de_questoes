%Data da última alteração:19/10/2016
\chapter{Probabilidade} 
	\section{Atividades Básicas}
		\begin{list}{\textbf{Questão \arabic{quest}.}}{\usecounter{quest}}
%define a margem da lista.	
%\setlength{\labelwidth}{-2mm} \setlength{\parsep}{0mm}
%\setlength{\topsep}{0mm} \setlength{\leftmargin}{-2mm}
\renewcommand{\labelenumi}{(\alph{enumi})}

		
		\item Lançando-se  um  dado  ideal,  qual  a  probabilidade  de  se  obter  um número menor que 4? 
		\item Uma  carta  é  retirada  ao  acaso  de  um  baralho de  52  cartas.  Qual  a probabilidade de ser uma dama? 
		\item Uma  carta  é  retirada  ao  acaso  de  um  baralho de  52  cartas.  Qual  a probabilidade de ser uma dama ou um rei? 
		\item Uma  urna  contém  6  bolas  brancas,  2  azuis  e  4  amarelas.  Qual  a probabilidade de sortear-se uma bola que não seja branca? 
		\item Numa urna há 20 bolas numeradas de 1 a 20. Uma bola é retirada ao acaso. Qual a probabilidade de ela ser um número ímpar? 
		\item Numa urna há 20 bolas numeradas de 1 a 20. Uma bola é retirada ao acaso. Qual a probabilidade de ela ser um número múltiplo de 3? 
		\item Numa urna há 20 bolas numeradas de 1 a 20. Uma bola é retirada ao acaso. Qual a probabilidade de ela ser um número divisível por 2 e 3? 
		\item Numa urna há 20 bolas numeradas de 1 a 20. Uma bola é retirada ao acaso. Qual a probabilidade de ela ser um número primo? 
		\item Um  grupo  de  amigos  organiza  uma  loteria  cujos  bilhetes  são formados  por  4  algarismos  distintos.  Qual  é  a  probabilidade  de uma  pessoa,  possuidora  dos  bilhetes  1387  e  7502,  ser  premiada, sendo que nenhum bilhete tem como algarismo inicial o zero? 
		\item Lançando-se  dois  dados  simultaneamente,  qual  a  probabilidade de ocorrerem números iguais? 
		\item Jogando-se dois dados simultaneamente, qual a probabilidade de se obter um resultado par na soma das faces? 
		\item Um número é escolhido ao acaso entre os 100 inteiros, de 1 a 100. Qual é a probabilidade do número ser múltiplo de 11? 
		\item Dentre 5 pessoas, será escolhida, por sorteio, uma comissão de 3 membros.  Qual  a  probabilidade  de  que  uma  determinada  pessoa venha a figurar na comissão? 
		\item Se  num  grupo  de  15  homens  e  5  mulheres  sorteamos  3  pessoas para  formarem  uma  comissão,  qual  a  probabilidade  de  que  ela seja formada por 2 homens e 1 mulher? 
		\item Numa urna são depositadas 9 etiquetas numeradas de 1 a 9. Três etiquetas são sorteadas (sem reposição). Qual a probabilidade de que os números sorteados sejam consecutivos? 
		\item Considere   as   24   permutações,   sem   repetição, que podemos formar  com  os  algarismos  3,4,5  e 7.  Se  uma  delas  é  escolhida  ao acaso,  determine  a  probabilidade  de  ser  um  número  maior  que 5000. 
		\item Considere   as   24   permutações,   sem   repetição, que podemos formar  com  os  algarismos  3,4,5  e 7.  Se  uma  delas  é  escolhida  ao acaso, determine a probabilidade de ser um número ímpar. 
		\item Considere   as   24   permutações,   sem   repetição, que podemos formar  com  os  algarismos  3,4,5  e7.  Se  uma  delas  é  escolhida  ao acaso, determine a probabilidade de ser um número par. 
		\item Numa escola de 1200 alunos, 550 gostam de rock, 230 apenas de samba, e 120 de samba e rock. Escolhendo-se um aluno ao acaso, qual a probabilidade de ele gostar de samba ou rock?
		\item Numa   urna,   existem   10   bolas   coloridas.   As   brancas   estão numeradas  de  1  a  6  e  as  vermelhas  de  7  a  10.  Retirando-se  uma bola, qual a probabilidade de ela ser branca ou de seu número ser maior que 7? 
		\item Uma carta é retirada ao acaso de um baralho de 52 cartas. Qual a probabilidade de ela ser de ouros ou ser rei. 
		\item Uma carta é retirada ao acaso de um baralho de 52 cartas. Qual a probabilidade de ela ser preta ou ser figura? 
		\item Uma carta é retirada ao acaso de um baralho de 52 cartas. Qual a probabilidade de ela não ser figura ou ser um ás? 
		\item Uma  caixa  contém  1000  bolas  numeradas  de  1 a  1000.  Qual  a probabilidade   de   se   tirar,   ao   acaso,   uma   bola   contendo   um número par ou um número de 2 algarismos? 
		\item Num grupo, 50 pessoas pertencem a um clube A, 70 a um clube B, 30 a um clube C, 20 pertencem aos clubes A e B, 22 aos clubes A e C, 18 aos clubes B e C e 10 pertencem aos três clubes. Escolhida, ao acaso, uma das pessoas presentes, qual a probabilidade de ela pertencem somente ao clube C? 
		\item Numa urna temos bolas brancas, amarelas, vermelhas e pretas. O número de bolas amarelas é o dobro do número de bolas brancas e  o  de  bolas  vermelhas,  o  triplo.  Qual  a  probabilidade  de  ocorrer uma bola preta, sabendo-se que o número de pretas é o dobro de amarelas? 
		\item Uma estação meteorológica informa: "Hoje a probabilidade de não chover   é   55\%,   a   probabilidade   de   fazer   frio   é   35\% e a probabilidade de chover ou fazer frio é 80\%. Qual a probabilidade de não chover e não fazer frio? 
		\item No problema anterior, qual a probabilidade de chover? 
		\end{list}		
	\section{Atividades complementares}
		\begin{list}{\textbf{Questão \arabic{quest}.}}{\usecounter{quest}}
%define a margem da lista.	
%\setlength{\labelwidth}{-2mm} \setlength{\parsep}{0mm}
%\setlength{\topsep}{0mm} \setlength{\leftmargin}{-2mm}
\renewcommand{\labelenumi}{(\alph{enumi})}

			\item  Uma bola será retirada de uma sacola contendo 5 bolas verdes e 7 bolas amarelas. Qual a probabilidade desta bola ser verde?

 
			\item Três moedas são lançadas ao mesmo tempo. Qual é a probabilidade de as três moedas caírem com a mesma face para cima?

 
			\item Um casal pretende ter filhos. Sabe-se que a cada mês a probabilidade da mulher engravidar é de 20\%. Qual é a probabilidade dela vir a engravidar somente no quarto mês de tentativas?

 
			\item Um credor está à sua procura. A probabilidade dele encontrá-lo em casa é 0,4. Se ele fizer 5 tentativas, qual a probabilidade do credor lhe encontrar uma vez em casa?

 
			\item Em uma caixa há 2 fichas amarelas, 5 fichas azuis e 7 fichas verdes. Se retirarmos uma única ficha, qual a probabilidade dela ser verde ou amarela?

 
			\item Alguns amigos estão em uma lanchonete. Sobre a mesa há duas travessas. Em uma delas há 3 pastéis e 5 coxinhas. Na outra há 2 coxinhas e 4 pastéis. Se ao acaso alguém escolher uma destas travessas e também ao acaso pegar um dos salgados, qual a probabilidade de se ter pegado um pastel?

 
			\item O jogo de dominó é composto de peças retangulares formadas pela junção de dois quadrados. Em cada quadrado há a indicação de um número, representado por uma certa quantidade de bolinhas, que variam de nenhuma a seis. O número total de combinações possíveis é de 28 peças. Se pegarmos uma peça qualquer, qual a probabilidade dela possuir ao menos um 3 ou 4 na sua face?

 
			\item Em uma caixa há 4 bolas verdes, 4 azuis, 4 vermelhas e 4 brancas. Se tirarmos sem reposição 4 bolas desta caixa, uma a uma, qual a probabilidade de tirarmos nesta ordem bolas nas cores verde, azul, vermelha e branca?

 
			\item Em uma escola de idiomas com 2000 alunos, 500 alunos fazem o curso de inglês, 300 fazem o curso de espanhol e 200 cursam ambos os cursos. Selecionando-se um estudante do curso de inglês, qual a probabilidade dele também estar cursando o curso de espanhol?

 
			\item De uma sacola contendo 15 bolas numeradas de 1 a 15 retira-se uma bola. Qual é a probabilidade desta bola ser divisível por 3 ou divisível por 4?
			
%Contúdo acrescentado por Arthur Astolfi Tótola em 19/10/2016
%==============================================================================================================================	
%==============================================================================================================================		
\item Os números naturais de 1 a 10 foram escritos, um a um, sem repetição, em dez bolas de pingue-pongue. Se duas delas forem escolhidas ao acaso, o valor mais provável da soma dos números sorteados é igual a:
\begin{multicols}{5}
\begin{enumerate}[a)]
	\item 9
	\item 10
	\item 11 x
	\item 12
	\item 13
\end{enumerate}
\end{multicols}

\item Uma moeda é viciada, de forma que as caras são três vezes mais prováveis de aparecer do que as coroas. Determine a probabilidade de num lançamento sair coroa.
\begin{multicols}{5}
\begin{enumerate}[a)]
	\item 25\% x
	\item 50\%
	\item 35\%
	\item 70\%
	\item 20\%
\end{enumerate}
\end{multicols}

\item Um cartão é retirado aleatoriamente de um conjunto de 50 cartões numerados de 1 a 50. Determine a probabilidade do cartão retirado ser de um número primo.
\begin{multicols}{5}
\begin{enumerate}[a)]
	\item 1/3 
	\item 1/5
	\item 2/5
	\item 3/10  x
	\item 7/10
\end{enumerate}
\end{multicols}

\item Escolhem-se ao acaso dois números naturais distintos, de 1 a 20. Qual a probabilidade de que o produto dos números escolhidos seja ímpar?
\begin{multicols}{5}
\begin{enumerate}[a)]
	\item 9/38  x
	\item 1/2
	\item 9/20  
	\item 1/4
	\item 8/25
\end{enumerate}
\end{multicols}

\item Uma carta é retirada de um baralho comum, de 52 cartas, e, sem saber qual é a carta, é misturada com as cartas de um outro baralho idêntico ao primeiro. Retirando, em seguida, uma carta do segundo baralho, a probabilidade de se obter uma dama é:
\begin{multicols}{5}
\begin{enumerate}[a)]
	\item 3/51
	\item 5/53
	\item 5/676
	\item 1/13  x
	\item 5/689 
\end{enumerate}
\end{multicols}

\item Três pessoas A, B e C vão participar de um concurso num programa de televisão. O apresentador faz um sorteio entre A e B e, em seguida, faz um sorteio, para decidir quem iniciará o concurso. Se em cada sorteio as duas pessoas têm a mesma "chance" de ganhar, qual é a probabilidade de A iniciar o concurso?
\begin{multicols}{5}
\begin{enumerate}[a)]
	\item 12,5\%
	\item 25\%  x
	\item 50\%
	\item 75\%
	\item 95\%
\end{enumerate}
\end{multicols}

\item (UFJF) Uma prova de certo concurso contém 5 questões com 3 alternativas de resposta para cada uma, sendo somente uma dessas alternativas a resposta correta. Em cada questão, o candidato deve escolher uma das três alternativas como resposta. Certo candidato que participa desse concurso decidiu fazer essas escolhas aleatoriamente. A probabilidade, desse candidato, escolher todas as respostas corretas nessa prova é igual a:
\begin{multicols}{5}
\begin{enumerate}[a)]
	\item 3/5
	\item 1/3
	\item 1/15
	\item 1/125
	\item 1/243  x
\end{enumerate}
\end{multicols}

\item (UFJF) Um soldado do esquadrão anti-bombas tenta desativar certo artefato explosivo que possui 5 fios expostos. Para desativá-lo, o soldado precisa cortar 2 fios específicos, um de cada vez, em uma determinada ordem. Se cortar um fio errado ou na ordem errada, o artefato explodirá. Se o soldado escolher aleatoriamente 2 fios para cortar, numa determinada ordem, a probabilidade do artefato não explodir ao cortá-los é igual a:
\begin{multicols}{5}
\begin{enumerate}[a)]
	\item 2/25
	\item 1/20  x
	\item 2/5
	\item 1/10
	\item 9/20
\end{enumerate}
\end{multicols}

\item (PUC) De sua turma de 30 alunos, é escolhida uma comissão de 3 representantes. Qual a probabilidade de você fazer parte da comissão?
\begin{multicols}{5}
\begin{enumerate}[a)]
	\item 1/10  x
	\item 1/12
	\item 5/24
	\item 1/3
	\item 2/9
\end{enumerate}
\end{multicols}

\item (FGV) Um jogador aposta que, em três lançamentos de uma moeda honesta, obterá duas caras e uma coroa. A probabilidade de que ele ganhe a aposta é:
\begin{multicols}{5}
\begin{enumerate}[a)]
	\item 1/3
	\item 2/3
	\item 1/8
	\item 3/8  x
	\item 5/8
\end{enumerate}
\end{multicols}

\item (UFV) Os bilhetes de uma rifa são numerados de 1 a 100. A probabilidade do bilhete sorteado ser um número maior que 40 ou número par é:
\begin{multicols}{5}
\begin{enumerate}[a)]
	\item 60\%
	\item 70\% 
	\item 80\%
	\item 90\%
	\item 50\%
\end{enumerate}
\end{multicols}

\item (PUC) Considere uma família numerosa tal que:

• cada filho do sexo masculino tem um número de irmãs igual ao dobro do número de irmãos;

• cada filho do sexo feminino tem um número de irmãs igual ao de irmãos acrescido de 2 unidades;

Ao escolher-se ao acaso 2 filhos dessa família, a probabilidade de eles serem de sexos opostos é:
\begin{multicols}{5}
\begin{enumerate}[a)]
	\item 4/13
	\item 20/39  x
	\item 7/12
	\item 11/13
	\item 11/12 
\end{enumerate}
\end{multicols}

\item (FUVEST) Um arquivo de escritório possui 4 gavetas, chamadas a, b, c, d. Em cada gaveta cabem no máximo 5 pastas. Uma secretária guardou, no acaso,18 pastas nesse arquivo. Qual é a probabilidade de haver exatamente 4 pastas na gaveta a?
\begin{multicols}{5}
\begin{enumerate}[a)]
	\item 3/10  x
	\item 1/10
	\item 3/20
	\item 1/20 
	\item 1/30
\end{enumerate}
\end{multicols}

\item (UFSCAR-SP) Gustavo e sua irmã Caroline viajaram de férias para cidades distintas. Os pais recomendam que ambos telefonem quando chegarem ao destino. A experiência em férias anteriores mostra que nem sempre Gustavo e Caroline cumprem esse desejo dos pais. A probabilidade de Gustavo telefonar é 0,6 e a probabilidade de Caroline telefonar é 0,8. A probabilidade de pelo menos um dos filhos contatar os pais é:
\begin{multicols}{5}
\begin{enumerate}[a)]
	\item 0,20
	\item 0,48
	\item 0,64
	\item 0,86
	\item 0,92
\end{enumerate}
\end{multicols}

\item Numa caixa são colocados vários cartões, alguns amarelos, alguns verdes e os restantes pretos. Sabe-se que 50\% dos cartões são pretos, e que, para cada três cartões verdes, há 5 cartões pretos. Retirando-se ao acaso um desses cartões, a probabilidade de que este seja amarelo é de:
\begin{multicols}{5}
\begin{enumerate}[a)]
	\item 10\%
	\item 15\%
	\item 20\% x
	\item 25\%
	\item 40\%
\end{enumerate}
\end{multicols}

\item (UFSP) Tomam-se 20 bolas idênticas (a menos da cor), sendo 10 azuis e dez brancas. Acondicionam-se as azuis numa urna A e as brancas numa urna B. Transportam-se 5 bolas da urna B para a urna A e, em seguida, transportam-se 5 bolas da urna A para a urna B. Sejam p a probabilidade de se retirar ao acaso uma bola branca da urna A e q a probabilidade de se retirar ao acaso uma bola azul da urna B. Então:
\begin{multicols}{5}
\begin{enumerate}[a)]
	\item p = q  x
	\item p = 2/10  e  q = 3/10
	\item p = 3/10  e  q = 2/10
	\item p = 1/10  e  q = 4/10
	\item p = 4/10  e  q = 1/10
\end{enumerate}
\end{multicols}

\item (PUC) Serão sorteados 4 prêmios iguais entre os 20 melhores alunos de um colégio, dentre os quais estão Tales e Euler. Se cada aluno pode receber apenas um prêmio, a probabilidade de que Tales ou Euler façam parte do grupo sorteado é:
\begin{multicols}{5}
\begin{enumerate}[a)]
	\item 3/95
	\item 1/19
	\item 3/19
	\item 7/19  x
	\item 38/95
\end{enumerate}
\end{multicols}

\item (UNESP) Para uma partida de futebol, a probabilidade de o jogador R não ser escalado é 0,2 e a probabilidade de o jogador S ser escalado é 0,7. Sabendo que a escalação de um deles é independente da escalação do outro, a probabilidade de os dois jogadores serem escalados é:
\begin{multicols}{5}
\begin{enumerate}[a)]
	\item 0,06
	\item 0,14
	\item 0,24
	\item 0,56 x
	\item 0,72 
\end{enumerate}
\end{multicols}

\item (FUVEST) Dois triângulos congruentes, com lados coloridos, são indistinguíveis se podem ser sobrepostos de tal modo que as cores dos lados coincidentes sejam as mesmas. Dados dois triângulos equiláteros congruentes, cada um de seus lados é pintado com uma cor escolhida dentre duas possíveis, com igual probabilidade. A probabilidade de que esses triângulos sejam indistinguíveis é de:
\begin{multicols}{5}
\begin{enumerate}[a)]
	\item 1/2
	\item 3/4
	\item 9/16 
	\item 5/16  x
	\item 15/32
\end{enumerate}
\end{multicols}

\item (FGV-SP) Um recipiente contém 4 balas de hortelã, 5 de morango e 3 de anis. Se duas balas forem sorteadas sucessivamente e sem reposição, a probabilidade de que sejam de mesmo sabor é:
\begin{multicols}{5}
\begin{enumerate}[a)]
	\item 18/65 
	\item 19/66 x
	\item 20/67
	\item 21/68
	\item 22/69
\end{enumerate}
\end{multicols}

\item (FGV-SP) A área da superfície da Terra é aproximadamente 510 milhões de $km^2$. Um satélite artificial dirige-se aleatoriamente para a Terra. Qual a probabilidade de ele cair numa cidade cuja superfície tem área igual a 102 $km^2$?
\begin{multicols}{5}
\begin{enumerate}[a)]
	\item $2\cdot 10^{–9}$
	\item $2cdot 10^{–8}$
	\item $2cdot 10^{–7}$
	\item $2cdot 10^{–6} \ \ x$
	\item $2cdot 10^{–5}$
\end{enumerate}
\end{multicols}

\item (PUC) De uma turma de 30 alunos, é escolhida uma comissão de 3 representantes. Qual a probabilidade de você fazer parte da comissão?
\begin{multicols}{5}
\begin{enumerate}[a)]
	\item 1/10 x
	\item 1/12 
	\item 5/24
	\item 1/3
	\item 2/9
\end{enumerate}
\end{multicols}

\item As cartas de um baralho são amontoadas aleatoriamente. Qual é a probabilidade de a carta de cima ser de copas e a de baixo também? O baralho é formado por 52 cartas de 4 naipes diferentes (13 de cada naipe).
\begin{multicols}{5}
\begin{enumerate}[a)]
	\item 1/17  x
	\item 1/25
	\item 1/27
	\item 1/36
	\item 9/45
\end{enumerate}
\end{multicols}

\item Numa maternidade, aguarda-se o nascimento de três bebês. Se a probabilidade de que cada bebê seja menino é igual à probabilidade de que cada bebê seja menina, a probabilidade de que os três bebês sejam do mesmo sexo é:
\begin{multicols}{5}
\begin{enumerate}[a)]
	\item 1/2
	\item 1/3 
	\item 1/4 x
	\item 1/6  
	\item 1/8
\end{enumerate}
\end{multicols}

\item Considere dois dados, cada um deles com seis faces, numeradas de 1 a 6. Se os dados são lançados ao acaso, a probabilidade de que a soma dos números sorteados seja 5 é:
\begin{multicols}{5}
\begin{enumerate}[a)]
	\item 1/15  x
	\item 2/21
	\item 1/12
	\item 1/11
	\item 1/9
\end{enumerate}
\end{multicols}

\item Em um município, uma pesquisa revelou que 5\% dos domicílios são de pessoas que vivem sós e, dessas, 52\% são homens. Com base nessas informações, escolhendo-se ao acaso uma pessoa desse município, a probabilidade de que ela viva só e seja mulher é igual a:
\begin{multicols}{5}
\begin{enumerate}[a)]
	\item 0,530
	\item 0,240
	\item 0,053
	\item 0,048
	\item 0,024  x
\end{enumerate}
\end{multicols}

\item  Dois números inteiros são selecionados aleatoriamente de 1 a 9. Se a soma é par, a probabilidade de os números serem ímpares é:
\begin{multicols}{5}
\begin{enumerate}[a)]
	\item 0,3
	\item 2/5
	\item 1/5
	\item 0,32
	\item 5/8 x 
\end{enumerate}
\end{multicols}

\item Seja o conjunto A = \{1,2,3,4,5,6\} . Escolhendo-se três elementos distintos de A , a probabilidade de que eles representem as medidas dos lados de um triângulo é:
\begin{multicols}{5}
\begin{enumerate}[a)]
	\item 0,35
	\item 0,45
	\item 0,55
	\item 0,65 
	\item 0,25
\end{enumerate}
\end{multicols}

\item Inteiramente ao acaso, 14 alunos dividiram-se em 3 grupos de estudos. O primeiro, para estudar Matemática, o segundo, Física, e o terceiro, Química. Se em cada um dos grupos há pelo menos 4 alunos, a probabilidade de haver exatamente 5 alunos no grupo que estuda Matemática é de:
\begin{multicols}{5}
\begin{enumerate}[a)]
	\item 1/3  x
	\item 2/3
	\item 3/4
	\item 5/6
	\item 1
\end{enumerate}
\end{multicols}

\item Para acessar o sistema de computadores da empresa, cada funcionário digita sua senha pessoal, formada por 4 letras distintas do nosso alfabeto (que possui 23 letras), numa ordem preestabelecida. Certa vez, um funcionário esqueceu a respectiva senha, lembrando apenas que ela começava com X e terminava com F. A probabilidade de ele ter acertado a senha ao acaso, numa única tentativa, é:
\begin{multicols}{4}
\begin{enumerate}[a)]
	\item 1/326
	\item 1/529
	\item 1/253
	\item 1/420
\end{enumerate}
\end{multicols}

\item  "Blocos Lógicos" é uma coleção de peças utilizada no ensino de Matemática. São 48 peças construídas combinando-se 3 cores (azul, vermelha e amarela), 4 formas (triangular, quadrada, retangular e circular), 2 tamanhos (grande e pequeno) e 2 espessuras (grossa e fina). Cada peça tem apenas uma cor, uma forma, um tamanho e uma espessura. Se uma criança pegar uma peça, aleatoriamente, a probabilidade dessa peça ser amarela e grande é:
\begin{multicols}{5}
\begin{enumerate}[a)]
	\item 1/12
	\item 1/6  x
	\item 1/3
	\item 1/2
\end{enumerate}
\end{multicols}

\item José, João, Manoel, Lúcia, Maria e Ana foram ao cinema e sentaram-se lado a lado, aleatoriamente, numa mesma fila. A probabilidade de José ficar entre Ana e Lúcia (ou Lúcia e Ana), lado a lado, é:
\begin{multicols}{5}
\begin{enumerate}[a)]
	\item 1/2
	\item 14/15
	\item 1/30
	\item 1/15  x
\end{enumerate}
\end{multicols}
%==============================================================================================================================
\end{list}
	
		
	