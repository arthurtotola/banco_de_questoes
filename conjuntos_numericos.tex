\chapter{Conjuntos Numéricos}
\section{Atividade Elementares}
	\begin{list}{\textbf{Questão \arabic{quest}.}}{\usecounter{quest}}
%define a margem da lista.	
%\setlength{\labelwidth}{-2mm} \setlength{\parsep}{0mm}
%\setlength{\topsep}{0mm} \setlength{\leftmargin}{-2mm}
\renewcommand{\labelenumi}{(\alph{enumi})}

	\item Seja $A = \{ 1, \{2\}, \{1,2\} \}$. Considere as afirmações:
		\begin{multicols}{4}
		\newcounter{Romanos}
		\begin{list}{\Roman{Romanos} -}{\usecounter{Romanos}}
			\item $1 \displaystyle \in A$
			\item $2 \displaystyle \in A$
			\item $\displaystyle \varnothing \subset A$
			\item $\{1,2\} \displaystyle \subset A$
		\end{list}
		\end{multicols}
		Estão corretas as afirmações:
		\begin{multicols}{5}
		\begin{enumerate}
			\item I e II
			\item I e III
			\item III e IV	
			\item III
			\item I
		\end{enumerate}
		\end{multicols}

		\item Sabendo que A = {1, 2, 3, 4}, B = {4, 5, 6} e C = {1, 6, 7, 8, 9}, podemos afirmar que o conjunto $(A \displaystyle \cap B) \displaystyle \cup C$ é:
		\begin{multicols}{4}
		\begin{enumerate}
			\item \{1, 4\}
			\item \{1, 4, 6, 7\}
			\item \{1, 4, 5, 6\}	
			\item \{1, 4, 6, 7, 8, 9\}			
		\end{enumerate}
		\end{multicols}

		\item José Carlos e Marlene são os pais de Valéria. A família quer viajar nas férias de julho. José Carlos conseguiu tirar suas férias na fábrica do dia 2 ao dia 28. Marlene obteve licença no escritório de 5 a 30. As férias de Valéria na escola vão de 1 a 25. Durante quantos dias a família poderá viajar sem faltar as suas obrigações?
		\begin{multicols}{4}
		\begin{enumerate}
			\item 19
			\item 20
			\item 21 	
			\item  22			
		\end{enumerate}
		\end{multicols}

		\item (UNESP) Numa classe de 30 alunos, 16 gostam de Matemática e 20 gostam de História. O número de alunos desta classe que gostam de Matemática e História é:
		\begin{multicols}{3}
		\begin{enumerate}
			\item exatamente 16
			\item exatamente 10
			\item no máximo 6	
			\item no mínimo 6
			\item exatamente 18
		\end{enumerate}
		\end{multicols}

		\item (PUC) Numa pesquisa de mercado, verificou-se que 15 pessoas utilizam pelo menos um dos produtos A ou B. Sabendo que 10 destas pessoas não usam o produto B e que 2 destas pessoas não usam o produto A, qual é o número de pessoas que utilizam os produtos A e B?
		\begin{multicols}{4}
		\begin{enumerate}
			\item 2
			\item 3
			\item 4	
			\item 5 
		\end{enumerate}
		\end{multicols}
%===================================================================================================
	\item Qual é o conjunto dos números pares maiores que 50 e menores que 200?

	\item Qual é o conjunto das consoantes da palavra coco?

	\item Quantos elementos possui o conjunto \{3, 33, 333, 3333\}?

	\item Qual é o conjunto formado pelos números pares do número 31657?

	\item Faça um diagrama que satisfaça as seguintes condições: $2\in A, 4\notin A, 3\in A\cap B, A\cup B =\{2, 3, 4\}$.

	\item Como é representado um conjunto vazio?

	\item Seja o conjunto A = \{3, 4, 5, 6, 7, 8, 9\}, determine 3 subconjuntos de A, ou seja, 3 conjuntos que estejam contidos em A.

	\item Sabendo que A = \{0, 1, 2, ..., 98, 99\}, B = \{1, 2, 10, 12\} e C = \{10, 11, 12, ..., 98, 99\}, podemos afirmar que:
	\begin{enumerate}
	\begin{multicols}{4}
		\item $A\subset B$
		\item $B\subset C$
		\item $C\subset A$
		\item $A\subset C$
	\end{multicols}
	\end{enumerate}



	\item Sendo A = \{1, 2, 3, 4, 5\}, B = \{3, 4, 5, 6, 7\} e C = \{5, 6, 7, 8, 9\}, determine:
	\begin{enumerate}
	\begin{multicols}{4}
		\item $A\cup B$
		\item $A\cup C$
		\item $B\cup C$
		\item $A\cup B\cup C$
		\item $A\cap B$
		\item $A\cap C$	
		\item $B\cap C$
		\item $A\cap B\cap C$
	\end{multicols}
	\end{enumerate}
	
	\item Faça um diagrama que represente os conjuntos A, B e C da questão anterior.

	\item Quando temos $A\cap B = \varnothing$, dizemos que A e B são disjuntos. Escreva dois conjuntos, A e B, de modo que sejam disjuntos.

	\item Se o conjunto A tem 7 elementos, o conjunto B, 4 elementos e $A\cap B$ tem 1 elemento, quantos elementos terá $A\cup B$?

	\item Numa pesquisa em que foram ouvidas crianças, constatou-se que:
	\begin{itemize}
		\item 15 crianças gostavam de refrigerante.
		\item 25 crianças gostavam de sorvete
		\item 5 crianças gostavam de refrigerante e de sorvete
	\end{itemize}
	
Quantas crianças foram pesquisadas?

	\item Foram instaladas 66 lâmpadas para iluminar as ruas A e B, que se cruzam. Na rua A foram colocadas 40 lâmpadas e na rua B 30 lâmpadas. Quantas lâmpadas foram instaladas no cruzamento?

	\item Numa concentração de atletas há 42 que jogam basquetebol, 28 voleibol e 18 voleibol e basquetebol, simultaneamente. Qual é o número de atletas na concentração?

	\item Uma atividade com duas questões foi aplicada em uma classe de 40 alunos. Os resultados apontaram que 20 alunos haviam acertado as duas questões, 35 acertaram a primeira questão e 25, a segunda. Faça o diagrama e calcule o percentual de alunos que acertou apenas uma questão?

	\item Uma pesquisa de mercado foi realizada para verificar a audiência de três programas de televisão, 1200 famílias foram entrevistadas e os resultados obtidos foram os seguintes: 370 famílias assistem ao programa A, 300 ao programa B e 360 ao programa C. Desse total, 100 famílias assistem aos programas A e B, 60 aos programas B e C, 30 aos programas A e C e 20 famílias aos 3 programas.Com base nesses dados, determine:
	\begin{enumerate}
	\item quantas famílias não assistem a nenhum dos 3 programas?
	\item quantas famílias assistem ao programa A e não assistem ao programa C?
	\item qual o programa de maior fidelidade, ou seja, cujos espectadores assistem somente a esse programa?
	\end{enumerate}
	
	\end{list}