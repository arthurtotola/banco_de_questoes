\chapter{Equações do Segundo Grau}
	\section{Atividade Básicas}
	\begin{list}{\textbf{Questão \arabic{quest}.}}{\usecounter{quest}}
%define a margem da lista.	
%\setlength{\labelwidth}{-2mm} \setlength{\parsep}{0mm}
%\setlength{\topsep}{0mm} \setlength{\leftmargin}{-2mm}
\renewcommand{\labelenumi}{(\alph{enumi})}

	\item Identifique os coeficientes de cada equação e diga se ela é completa ou não:
		\begin{multicols}{2}
		\begin{enumerate}
		\item $5x^2 - 3x - 2 = 0$                                                 
		\item $3x^2  + 55 = 0$                                                     
		\item $x~2 - 6x = 0$                                               
		\item $x^2 - 10x + 25 = 0$
		\end{enumerate}	
		\end{multicols}		
	\item Resolva as equações do segundo grau.
	\begin{multicols}{3}
	\begin{enumerate}
		\item $x^2 - 5x + 6 = 0$
		\item $x^2 - 8x + 12 = 0$
		\item $x^2 + 2x - 8 = 0$
		\item $x^2 - 5x + 8 = 0$
		\item $2x^2 - 8x + 8 = 0$  
		\item $x^2 - 4x - 5 = 0$
		\item $-x^2 + x + 12 = 0$  
		\item $-x^2 + 6x - 5 = 0$ 
		\item $6x^2 + x - 1 = 0$
		\item $3x^2 - 7x + 2 = 0$
		\item $2x^2 - 7x = 15$
		\item $4x^2 + 9 = 12x$
		\item $x^2 = x + 12$
		\item $2x^2 = -12x - 18$
		\item $x^2 + 9 = 4x$
		\item $25x^2 = 20x - 4$
		\item $2x = 15 - x^2$
		\item $x^2 + 3x - 6 = -8$
		\item $x^2 + x - 7 = 5$
		\item $4x^2 - x + 1 = x + 3x^2$
		\item $3x^2 + 5x = -x - 9 + 2x^2$
		\item $4 + x ( x - 4) = x$
		\item $x ( x + 3) - 40 = 0$
		\item $x^2 + 5x + 6 = 0$
		\item $x^2 - 7x + 12 = 0$
		\item $x^2 + 5x + 4 = 0$
	\end{enumerate}
	\end{multicols}
	\item Calcule as raízes das equações do segundo grau.
	\begin{multicols}{3}
	\begin{enumerate}	
		\item $7x^2 + x + 2 = 0$
		\item $x^2 - 18x + 45 = 0$
		\item $-x^2 - x + 30 = 0$
		\item $x^2 - 6x + 9 = 0$
		\item $(x + 3)^2 = 1$
		\item $(x - 5)^2 = 1$
		\item $(2x - 4)^2 = 0$
		\item $(x - 3)^2 = -2x^2$
		\item $x^2 + 3x - 28 = 0$
		\item $3x^2 - 4x + 2 = 0$
		\item $x^2 - 3 = 4x + 2$
	\end{enumerate}
	\end{multicols}
		
	\item Achar as raízes das equações: 
		\begin{multicols}{3}
		\begin{enumerate}
		\item $x^2 - x - 20 = 0$
		\item $x^2 - 3x -4 = 0$
		\item $x^2 - 8x + 7 = 0$
		\end{enumerate} 
		\end{multicols}
	\item Dentre os números $-2, 0, 1, 4$, quais deles são raízes da equação $x^2-2x-8= 0$?
	\item Determine quais os valores de $k$ para que a equação $2x^2 + 4x + 5k = 0$ tenha raízes reais e distintas.	
	\end{list}
	\section{Problemas Envolvendo Equações do Segundo Grau}
	\begin{list}{\textbf{Questão \arabic{quest}.}}{\usecounter{quest}}
%define a margem da lista.	
%\setlength{\labelwidth}{-2mm} \setlength{\parsep}{0mm}
%\setlength{\topsep}{0mm} \setlength{\leftmargin}{-2mm}
\renewcommand{\labelenumi}{(\alph{enumi})}

	\item A soma de um número com o seu quadrado é 90. Calcule esse numero.
	\item A soma do quadrado de um número com o próprio número é 12. Calcule esse numero.
	\item O quadrado menos o dobro de um número é igual a -1. Calcule esse número.
	\item A diferença entre o quadrado e o dobro de um mesmo número é 80. Calcule esse número.
	\item O quadrado de um número aumentado de 25 é igual a dez vezes esse número.         Calcule esse número.
	\item O quadrado menos o quádruplo de um numero é igual a 5. Calcule esse número.
	\item O quadrado de um número é igual ao produto desse número por 3, mais 18. Qual é esse numero?
	\item O dobro do quadrado de um número é igual ao produto desse numero por 7 menos 3. Qual é esse numero?
	\item O quadrado de um número menos o triplo do seu sucessivo é igual a 15. Qual é esse numero?
	\item Qual o número que somado com seu quadrado resulta em 56?
	\item Um numero ao quadrado mais o dobro desse número é igual a 35. Qual é esse número ?
	\item O quadrado de um número menos o seu triplo é igual a 40. Qual é esse número?
	\item Calcule um número inteiro tal que três vezes o quadrado desse número menos o dobro desse número seja igual a 40.
	\item Calcule um número inteiro e positivo tal que seu quadrado menos o dobro desse número seja igual a 48.
	\item O triplo de um número menos o quadrado desse número é igual a 2. Qual é esse número?
	\item Qual é o número , cujo quadrado mais seu triplo é igual a 40?
	\item O quadrado de um número diminuído de 15 é igual ao seu dobro. Calcule esse número
	\item Determine um número tal que seu quadrado diminuído do seu triplo é igual a 26.
	\item Se do quadrado de um número, negativo subtraímos 7, o resto será 42. Qual é esse número?
	\item A diferença entre o dobro do quadrado de um número positivo e o triplo desse número é 77. Calcule o número.
	\item Determine dois números ímpares consecutivos cujo produto seja 143.
	\item Um azulejista usou $2000$ azulejos quadrados e iguais para revestir $45m^2$ de parede. Qual é a medida do lado de cada azulejo?
	\end{list}